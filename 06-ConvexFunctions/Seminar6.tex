\documentclass[12pt,russian]{beamer}
\usepackage{mypres}
\usepackage[utf8]{inputenc}
\usepackage[english,russian]{babel}
\usepackage[T2A]{fontenc}
\newcommand{\relint}{\mathbf{relint}}
\newcommand{\inter}{\mathbf{int}}

\expandafter\def\expandafter\insertshorttitle\expandafter{%
  \insertshorttitle\hfill%
  \insertframenumber\,/\,\inserttotalframenumber}
\title[Семинар 6]{Методы оптимизации. \\
 Семинар 6. Выпуклые функции.}
\author{Александр Катруца}
\institute{Московский физико-технический институт,\\
Факультет Управления и Прикладной Математики} 
\date{10 октября 2016 г.}

\begin{document}
\begin{frame}
\maketitle
\end{frame}

\begin{frame}{Напоминание}
\begin{itemize}
\item Производная по скаляру
\item Производная по вектору
\item Производная по матрице
\item Производная сложной функции
\end{itemize}
\end{frame}

\begin{frame}{Определения функций}
\small
\begin{block}{Выпуклая функция}
Функция $f: X \subset \bbR^n \rightarrow \bbR$ называется выпуклой (строго выпуклой), если $X$~--- выпуклое множество и для $\forall \bx_1, \bx_2 \in X$ и $\alpha \in [0, 1] \; (\alpha \in (0, 1))$  выполнено:
\vspace{-4mm}
\[
f(\alpha \bx_1 + (1 - \alpha)\bx_2) \leq \; (<) \; \alpha f(\bx_1) + (1 - \alpha)f(\bx_2)
\]
\end{block}

\begin{block}{Вогнутая функция}
Функция $f$ вогнутая (строго вогнутая), если $-f$ выпуклая (строго выпуклая).
\end{block}

\begin{block}{Сильно выпуклая функция}
Функция $f: X \subset \bbR^n \rightarrow \bbR$ называется сильно  выпуклой с константой $\theta > 0$, если $X$~--- выпуклое множество и для $\forall \bx_1, \bx_2 \in X$ и $\alpha \in [0, 1]$  выполнено:
\vspace{-4mm}
\[
f(\alpha \bx_1 + (1 - \alpha)\bx_2) \leq \alpha f(\bx_1) + (1 - \alpha)f(\bx_2) - \theta \alpha (1 - \alpha) \| \bx_1 - \bx_2 \|_2^2
\]
\end{block}

\end{frame}

\begin{frame}{Определения множеств}
\begin{block}{Надграфик (эпиграф)}
Надграфиком функции $f$ называется множество $\text{epi}f = \{ (\bx, y) : \bx \in \bbR^n, \; y \in \bbR, \; y \geq f(\bx) \} \subset \bbR^{n+1}$
\end{block}

\begin{block}{Множество подуровней (множество Лебега)}
Множество подуровня функции $f$ называется следующее множество
\[
C_{\gamma} = \{ \bx | f(\bx) \leq \gamma \}.
\]
\end{block}
\end{frame}

\begin{frame}{Критерии выпуклости}
\begin{block}{Дифференциальный критерий первого порядка}

\end{block}

\begin{block}{Дифференциальный критерий второго порядка}

\end{block}

\begin{block}{Связь с надграфиком}

\end{block}

\begin{block}{Ограничение на прямую}

\end{block}
\end{frame}

\begin{frame}{Примеры}

\end{frame}

\begin{frame}{Неравенство Йенсена} 

\end{frame}

\begin{frame}{Примеры}

\end{frame}

\begin{frame}{Резюме}
\begin{itemize}
\item Выпуклая функция
\item Надграфик и множество подуровня функции
\item Критерии выпуклости функции
\item Неравенство Йенсена
\end{itemize}
\end{frame}
\end{document}