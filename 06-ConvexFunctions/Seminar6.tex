\documentclass[12pt,russian]{beamer}
\usepackage{mypres}
\usepackage[utf8]{inputenc}
\usepackage[english,russian]{babel}
\usepackage[T2A]{fontenc}
\newcommand{\relint}{\mathbf{relint}}
\newcommand{\inter}{\mathbf{int}}

\expandafter\def\expandafter\insertshorttitle\expandafter{%
  \insertshorttitle\hfill%
  \insertframenumber\,/\,\inserttotalframenumber}
\title[Семинар 6]{Методы оптимизации. \\
 Семинар 6. Выпуклые функции.}
\author{Александр Катруца}
\institute{Московский физико-технический институт,\\
Факультет Управления и Прикладной Математики} 
\date{10 октября 2016 г.}

\begin{document}
\begin{frame}
\maketitle
\end{frame}

\begin{frame}{Напоминание}
\begin{itemize}
\item Производная по скаляру
\item Производная по вектору
\item Производная по матрице
\item Производная сложной функции
\end{itemize}
\end{frame}

\begin{frame}{Определения}
\begin{block}{Выпуклая функция}

\end{block}

\begin{block}{Вогнутая функция}

\end{block}

\begin{block}{Сильно выпуклая функция}

\end{block}

\begin{block}{Надграфик}

\end{block}

\end{frame}

\begin{frame}{Критерии выпуклости}
\begin{block}{Дифференциальный критерий первого порядка}

\end{block}

\begin{block}{Дифференциальный критерий второго порядка}

\end{block}

\begin{block}{Связь с надграфиком}

\end{block}

\begin{block}{Ограничение на прямую}

\end{block}
\end{frame}

\begin{frame}{Примеры}

\end{frame}

\begin{frame}{Неравенство Йенсена} 

\end{frame}

\begin{frame}{Примеры}

\end{frame}

\begin{frame}{Резюме}
\begin{itemize}
\item Выпуклая функция
\item Надграфик и множество подуровня функции
\item Критерии выпуклости функции
\item Неравенство Йенсена
\end{itemize}
\end{frame}
\end{document}