\documentclass[12pt,russian]{beamer}
\usepackage{mypres}
\usepackage[utf8]{inputenc}
\usepackage[T2A]{fontenc}
\usepackage[russian]{babel}
\newcommand{\relint}{\mathbf{relint}}
\newcommand{\inter}{\mathbf{int}}

\expandafter\def\expandafter\insertshorttitle\expandafter{%
  \insertshorttitle\hfill%
  \insertframenumber\,/\,\inserttotalframenumber}
\title[Семинар 10]{Методы оптимизации. \\
 Семинар 10. Двойственность.}
\author{Александр Катруца}
\institute{Московский физико-технический институт,\\
Факультет Управления и Прикладной Математики} 
\date{14 октября 2016 г.}

\begin{document}
\begin{frame}
\maketitle
\end{frame}

\begin{frame}{Напоминание}
\begin{itemize}
\item Существование решения оптимизационной задачи 
\item Условия оптимальности для
\begin{itemize}
\item общей задачи оптимизации
\item задачи безусловной оптимизации
\item задачи оптимизации с ограничениями типа равенств
\item задачи оптимизации с ограничениями типа равенств и неравенств
\end{itemize}
\end{itemize}
\end{frame}

\begin{frame}{Обозначения}
\small
\begin{block}{Задача}
\vspace{-5mm}
\begin{equation*}
\begin{split}
& \min\limits_{x \in \mathcal{D}} f(x) = p^*\\
\text{s.t. } & g_i(x) = 0, \; i = 1,\ldots,m\\
& h_j(x) \leq 0, \; j = 1,\ldots, p
\end{split}
\end{equation*}
\end{block}

\begin{block}{Лагранжиан}
\vspace{-2mm}
\begin{equation*}
L(x, \blambda, \bmu) = f(x) + \sum\limits_{i=1}^m\lambda_i g_i(x) + \sum\limits_{j=1}^p \mu_j h_j(x)
\vspace{-2mm}
\end{equation*}
\end{block}

\begin{block}{Двойственные переменные}
Вектора $\bmu$ и $\blambda$ называются двойственными переменными.
\end{block}

\begin{block}{Двойственная функция}
Функция $g(\bmu, \blambda) = \inf\limits_{x\in \mathcal{D}} L(x, \blambda, \bmu)$ называется двойственной функцией Лагранжа.
\end{block}

\end{frame}

\begin{frame}{Свойства двойственной функции}
\small
\begin{block}{Вогнутость}
Двойственная функция является {\color{red}{вогнутой}} как инфимум аффинных функций по $(\bmu, \blambda)$ в независимости от того, является ли исходная задача выпуклой.
\end{block}

\begin{block}{Нижняя граница}
Для любого $\blambda$ и для $\bmu \geq 0$ выполнено $g(\bmu, \blambda) \leq p^*$.
\end{block}

\begin{block}{Двойственная задача}
\vspace{-5mm}
\begin{equation*}
\begin{split}
& \max g(\bmu, \blambda) = d^*\\
\text{s.t. } & \bmu \geq 0
\end{split}
\end{equation*}
\end{block}

\begin{block}{Зачем?}
\begin{itemize}
\vspace{-2mm}
\item Двойственная задача выпукла независимо от того, выпукла ли прямая
\vspace{-3mm}
\item Нижняя оценка может достигаться
\end{itemize}
\end{block}
\end{frame}

\begin{frame}{Примеры}
Найти двойственную функцию:
\begin{itemize}
\item Решение СЛУ минимальной нормы 
\vspace{-3mm}
\begin{equation*}
\begin{split}
& \min \| \bx\|^2_2\\
\text{s.t. } & \bA\bx = \mathbf{b}
\end{split}
\end{equation*}
\item Линейное программирование
\vspace{-3mm}
\begin{equation*}
\begin{split}
& \min \bc^{\T}\bx\\
\text{s.t. } & \bA\bx = \mathbf{b}\\
& \bx \geq 0
\end{split}
\end{equation*}
\item Задача разбиения
\vspace{-3mm}
\begin{equation*}
\begin{split}
& \min \bx^{\T}\bW\bx\\
\text{s.t. } & x^2_i = 1, \; i = 1,\ldots,n
\end{split}
\end{equation*}
\end{itemize}
\end{frame}

\begin{frame}{Слабая и сильная двойственность}

\end{frame}


\begin{frame}{Геометрическая интерпретация}

\end{frame}

\begin{frame}{Условия Каруша-Куна-Таккера}

\end{frame}

\begin{frame}{Механическая интерпретация}

\end{frame}

\begin{frame}{Примеры}

\end{frame}
\end{document}