\documentclass[12pt,russian]{beamer}
\usepackage{mypres}
\usepackage[utf8]{inputenc}
\usepackage[english,russian]{babel}
\usepackage[T2A]{fontenc}
\newcommand{\relint}{\mathbf{relint}}
\newcommand{\inter}{\mathbf{int}}

\expandafter\def\expandafter\insertshorttitle\expandafter{%
  \insertshorttitle\hfill%
  \insertframenumber\,/\,\inserttotalframenumber}
\title[Семинар 4]{Методы оптимизации. \\
 Семинар 4. Сопряжённые множества. Лемма Фаркаша.}
\author{Александр Катруца}
\institute{Московский физико-технический институт,\\
Факультет Управления и Прикладной Математики} 
\date{26 сентября 2016 г.}

\begin{document}
\begin{frame}
\maketitle
\end{frame}

\begin{frame}{Напоминание}
\begin{itemize}
\item Внутренность и относительная внутренность выпуклого множества
\item Проекция точки на множство
\item Отделимость выпуклых множеств
\item Опорная гиперплоскость
\end{itemize}
\end{frame}

\begin{frame}{Лемма Фаркаша}

\end{frame}

\begin{frame}{Примеры}

\end{frame}

\begin{frame}{Резюме}
\begin{itemize}
\item Сопряжённые множества
\item Свойства сопряжённых множеств
\item Лемма Фаркаша
\end{itemize}

\end{frame}

\end{document}