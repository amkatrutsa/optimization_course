\documentclass[12pt]{beamer}
\usepackage{../latex-sty/mypres}
\usepackage[utf8]{inputenc}
\usepackage[T2A]{fontenc}
\usepackage[english]{babel}

\expandafter\def\expandafter\insertshorttitle\expandafter{%
  \insertshorttitle\hfill%
  \insertframenumber\,/\,\inserttotalframenumber}
\title[Seminar 9]{Optimization methods. \\
 Seminar 9. Conjugate functions}
\author{Alexandr Katrutsa}
\institute{Moscow Institute of Physics and Technology\\
Department of Control and Applied Mathematics} 
\date{\today}

\begin{document}
\begin{frame}
\maketitle
\end{frame}

\begin{frame}{Reminder}
\begin{itemize}
\item Feasible direction cone
\item Tangent cone
\item Sharp extremum 
\end{itemize}
\end{frame}

\begin{frame}{Definition}
\begin{block}{Conjugacy again?}
\begin{itemize}
\item Previously we introduced conjugate (dual) sets and in particular conjugate cones
\item Today we consider conjugate (dual) functions
\item Further we will introduce dual (conjugate) optimization problem
\end{itemize}
\end{block}

\begin{block}{Definition}
A function $f^*: \bbR^n \rightarrow \bbR$ is called conjugate function of function $f$ and is defined as
\vspace{-4mm}
\[
f^*(\by) = \sup\limits_{\bx \in dom \; f} (\by^{\T}\bx - f(\bx)).
\vspace{-3mm}
\]
Domain of $f^*$ is a set of $\by$, such that the supremum is finite.

\end{block}
\end{frame}

\begin{frame}{Properties and interpretations}
\begin{itemize}
\item Conjugate function $f^*$ is always {\color{red}{convex}} as supremum of linear functions independently of convexity of $f$
\item Young-Fenchel inequality: 
\[
\by^{\T}\bx \leq f(\bx) + f^*(\by)
\]
\item If $f$ is differentiable, then $f^*(\by) = \nabla f^{\T}(\bx^*)\bx^* - f(\bx^*)$, where $\bx^*$ is a supremum point.
\item Geometrical interpretation
\end{itemize}
\end{frame}

\begin{frame}{Examples}
\begin{enumerate}
\item Linear function: $f(\bx) = \ba^{\T}\bx + b$
\item Negative entropy: $f(x) = x\log x$
\item Indicator function of the set $S$: $I_S(x) = 0$ iff $x \in S$
\item Norm: $f(\bx) = \|\bx\|$.
\item Squared norm: $f(\bx) = \frac{1}{2}\|\bx\|^2$
\end{enumerate}
\end{frame}

\begin{frame}{Recap}

\begin{itemize}
\item Conjugate functions
\item Young-Fenchel inequality and other properties
\item Examples
\end{itemize}

\end{frame}

\end{document}