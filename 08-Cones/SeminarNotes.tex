\documentclass[12pt]{article}
\usepackage{myart}
\usepackage[utf8]{inputenc}
\usepackage[russian]{babel}
\usepackage[T2A]{fontenc}
\usepackage{secdot}
\newtheorem{Def}{ Определение}
\begin{document}
\title{Семинар № 8 \\ 
<<Разные конусы и сопряжённые функции>>}
\author{Александр Катруца}
\date{\today}
\maketitle

\section{Разбор промежуточной контрольной}
Сначала несколько замечаний о прошедшей промежуточной контрольной.
\begin{enumerate}
\item Большинство написали все базовые определения и формулировки, и это уже неплохо. 
Те, кто не написал, пожалуйста, непременно выучите это к экзамену.
\item Почему-то почти никто не взялся сформулировать задачу проекции точки на вероятностный симплекс. 
Симплекс~--- это множество вида $\left \{ \bx \in \bbR_+^n \; | \; \sum\limits_{i=1}^n x_i = 1 \right \}$.
В условии он был назван вероятностным, потому что $\bx$ можно интерпретировать как дискретное вероятностное распределение.
И чтобы навести вас на определение симплекса, если вы его забыли.
Таким образом, задача нахождения проекции точки $\by$ на вероятностный симплекс формулируется следующим образом:
\begin{equation}
\begin{split}
&\min\limits_{\bx \in \bbR^n} \| \bx - \by \|_2\\
\text{s.t. } & x_i \geq 0\\
& \sum\limits_{i=1}^n x_i = 1
\end{split}
\end{equation}
и является выпуклой, так как целевая функция выпукла на выпуклой области определения~--- симплексе.
\item Далее, небольшое напоминание определения выпуклой задачи. 
\begin{Def}
Оптимизационная задача называется \emph{выпуклой}, если целевая функция является выпуклой на выпуклом допустимом множестве.
\end{Def}
\item В задаче про нормальный конус имелся в виду конус, с помощью которого опрелелялся условный субдифференциал. 
Но за ответ про конус, заданный некоторой нормой, я снижал совсем немного.
\item В задаче, где надо было проверить выпуклость множества, зависящего от параметра $t$, некоторые довольно мучительно, влоб, показали, что это множество пустое и поэтому выпуклое.
Однако ожидалось, что вы заметите, что при фиксированном $t$~---  это множество~--- шар, а значит выпукло.
А дальше надо было привести утверждение про выпуклость пересечения любого числа выпукых множеств.
Про этот приём я упоминал на семинаре.
\item В задаче про нахождение множества разделяющих плоскостей для заданной точки и множества, кажется, толкьо один человек правильно набросал путь решения. 
А именно надо было нарисовать картинку (специально задача дана в 2D) и понять, что для каждой точки, лежащей на оси ординат, между <<углом>> множества и данной точкой существует свой набор разделябщих гиперплоскостей.
То есть первый параметр~--- это точка на оси ординат, второй параметр~--- это множество угловых коэффициентов, которое зависит от первого параметра.
Далее нужно было аккуратно провести вычисления предельных положения разделяющих гиперплоскостей и записать ответ.   
\end{enumerate}

В целом, у меня осталось скорее положительное впечатление от вашей работы, однако я уже упоминал, что дальше контрольные будут оцениваться строже.
По крайней мере по отношению к тем, кто не может написать базовые определения курса.

\section{Конусы}

\section{Сопряжённые функции}

\end{document}