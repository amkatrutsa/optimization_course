\documentclass[12pt,russian]{beamer}
\usepackage{mypres}
\usepackage[utf8]{inputenc}
\usepackage[T2A]{fontenc}
\usepackage[russian]{babel}
\newcommand{\relint}{\mathbf{relint}}
\newcommand{\inter}{\mathbf{int}}

\expandafter\def\expandafter\insertshorttitle\expandafter{%
  \insertshorttitle\hfill%
  \insertframenumber\,/\,\inserttotalframenumber}
\title[Семинар 9]{Методы оптимизации. \\
 Семинар 9. Условия оптимальности.}
\author{Александр Катруца}
\institute{Московский физико-технический институт,\\
Факультет Управления и Прикладной Математики} 
\date{31 октября 2016 г.}

\begin{document}
\begin{frame}
\maketitle
\end{frame}

\begin{frame}{Напоминание}
\begin{itemize}
\item Конус возможных направлений
\item Касательный конус
\item Острый экстремум
\end{itemize}
\end{frame}

\begin{frame}{Мотивация}

\begin{block}{Вопрос 0}
Когда существует решение оптимизационной задачи?
\end{block}

\begin{block}{Вопрос 1}
Как проверить, что точка является решением оптимизационной задачи? 
\end{block}

\begin{block}{Вопрос 2}
Из каких условий можно найти решение оптимизационной задачи?
\end{block}
\end{frame}

\begin{frame}{Существование решения}
\begin{block}{Теорема Вейерштрасса}
Пусть $X \subset R^n$ компактное множество и пусть $f(x)$ непрерывная функция на $X$. 
Тогда точка глобального минимума функции $f (x)$ на $X$ существует.
\end{block}

Эта теорема гарантирует, что решение подавляющего большинства разумных задач существует.
 
\end{frame}

\begin{frame}{Условия оптимальности}
\begin{block}{Определение}
Условием оптимальности будем называть некоторое выражение, выполнимость которого даёт необходимое и (или) достаточное условие экстремума. 
\end{block}
Классы задач:
\begin{itemize}
\item Общая задача минимизации
\item Задача безусловной минимизации
\item Задача минимизации с ограничениями типа равенств
\item Задача минимизации с ограничениями типа равенств и неравенств
\end{itemize}
\end{frame}

\begin{frame}{Общая задача минимизации}

\begin{block}{Задача}
\[
f(x) \rightarrow \min\limits_{{\color{red}{x \in X}}}
\]
\end{block}

\begin{block}{Критерий оптимальности}
Пусть $f(x)$ определена на множестве $X \subset \bbR^n$.
Тогда 
\begin{enumerate}
\item если $x^*$ точка минимума $f(x)$ на $X$, то $\partial_X f(x^*) \neq \emptyset$ и $0 \in \partial_X f(x^*)$
\item если для некоторой точки $x^* \in X$ существует субдифференциал $\partial_X f(x^*)$ и $0 \in \partial_X f(x^*)$, то $x^*$~--- точка минимума $f(x)$ на $X$.
\end{enumerate}
\end{block}
Какие недостатки у приведённого критерия?

\end{frame}

\begin{frame}{Примеры}
\begin{itemize}
\item $\bx^{\T}\bx + \alpha \| \bx - 
\bc \|_2 \rightarrow \min\limits_{\bx \in \bbR^n}$, $\alpha > 0$
\item $\bx^{\T}\bx + \alpha \| \bc^{\T}\bx - 
b \|_2 \rightarrow \min\limits_{\bx \in \bbR^n}$, $\alpha > 0$
\item Ограничение на допустимое множество
\begin{equation*}
\begin{split}
\vspace{-4mm}
&(x + 2)^2 + |y + 3| \rightarrow \min\limits_{(x, y) \in \bbR^2}\\
\text{s.t. }& 8 + 2x - y \leq 0
\end{split}
\end{equation*}
\end{itemize}
\end{frame}

\begin{frame}{Задача безусловной минимизации}
Задача: $f(x) \rightarrow \min\limits_{\color{red}{x \in \bbR^n}}$.

\begin{block}{Критерий оптимальности для выпуклых функций}
Пусть $f(x)$ выпуклая функция на $\bbR^n$. 
Тогда точка $x^*$ решение задачи безусловной минимизации $\Leftrightarrow$ $0 \in \partial f(x^*)$.
\end{block}

\begin{block}{Следствие}
Если $f(x)$ выпукла и дифференцируема на $\bbR^n$.
Тогда точка $x^*$ решение задачи безусловной минимизации $\Leftrightarrow$ $\nabla f(x^*) = 0$.
\end{block}

\begin{block}{Достаточное условие для невыпуклых функций}
Пусть $f$ дважды дифференцируема на $\bbR^n$ и $x^*$ такая что $\nabla f(x^*) = 0$. 
Тогда если $\nabla^2 f(x^*) \succ 0$, то $x^*$ точка строгого локального минимума $f(x)$ на $\bbR^n$.  
\end{block}

\end{frame}

\begin{frame}{Примеры}
\begin{itemize}
\item $x_1e^{x_1} - (1 + e^{x_2})\cos x_2 \rightarrow \min$
\item Функция Розенброка: $(1 - x_1)^2 + \alpha \sum\limits_{i = 2}^n (x_i - x_{i-1})^2 \rightarrow \min$, $\alpha > 0$
\item $x^2_1 + x^2_2 - x_1x_2 + e^{x_1 + x_2} \rightarrow \min$
\end{itemize}
\end{frame}

\begin{frame}{{\small Задача минимизации с ограничениями типа равенств}}

\begin{block}{Задача}
\vspace{-3mm}
\begin{equation*}
\begin{split}
& f(x) \rightarrow \min\limits_{x \in \bbR^n} \\
\text{s.t. } & g_i(x) = 0, \; i = 1,\ldots, m 
\end{split}
\end{equation*}
\end{block}

\begin{block}{Лагранжиан}
\begin{equation*}
L(x, \blambda) = f(x) + \sum\limits_{i=1}^m\lambda_i g_i(x)
\end{equation*}
\end{block}

\begin{block}{Критерий оптимальности}
Пусть $f(x)$ и $g_i(x)$ дважды дифференцируемы в точке $x^*$ и непрерывно дифференцируемы в некоторой окрестности $x^*$.
Пусть также $\nabla_x L(x^*, \blambda) = 0$.
Тогда если $\nabla^2 L(x^*, \blambda) \succ 0$, то $x^*$~--- точка локального минимума.
\end{block}

\end{frame}

\begin{frame}{Примеры}
\begin{itemize}
\item $\sum\limits_{i=1}^n\alpha_i x^4_i \rightarrow \extr\limits_{\bx \in G}$, $G = \{\bx \in \bbR^n \; | \; \bc^{\T}\bx = 1 \}$, $\alpha_i > 0,\; c_i > 0$
\item $x_1 + 4x_2 + 9x_3 \rightarrow \extr\limits_{\bx \in G}, \; G = \left \{ \frac{1}{x_1} + \frac{1}{x_2} + \frac{1}{x_3} = 1 \right\}$
\item Примеры из задачника по матану на метод множителей Лагранжа
\end{itemize}
\end{frame}

\begin{frame}{{\small Задача минимизации с ограничениями типа равенств и неравенств}}

\begin{block}{Задача}

\end{block}

\begin{block}{Критерий оптимальности}

\end{block}

\end{frame}

\begin{frame}{Примеры}

\end{frame}

\end{document}