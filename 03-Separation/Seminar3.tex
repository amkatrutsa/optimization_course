\documentclass[12pt,russian]{beamer}
\usepackage{mypres}
\usepackage[utf8]{inputenc}
\usepackage[english,russian]{babel}
\usepackage[T2A]{fontenc}
\newcommand{\relint}{\mathbf{relint}}
\newcommand{\inter}{\mathbf{int}}

\expandafter\def\expandafter\insertshorttitle\expandafter{%
  \insertshorttitle\hfill%
  \insertframenumber\,/\,\inserttotalframenumber}
\title[Семинар 3]{Методы оптимизации. \\
 Семинар 3. Проекция точки на множество, отделимость, опорная гиперплоскость.}
\author{Александр Катруца}
\institute{Московский физико-технический институт,\\
Факультет Управления и Прикладной Математики} 
\date{19 сентября 2016 г.}

\begin{document}
\begin{frame}
\maketitle
\end{frame}

\begin{frame}{Напоминание}
\begin{itemize}
\item Афинная оболочка и множество
\item Выпуклая оболочка и множество
\item Коническая оболочка и выпуклый конус
\item Операции, сохраняющие выпуклость
\end{itemize}
\end{frame}

\begin{frame}{Внутренности множества}

\begin{block}{Внутренность множества}
Внутренность множества $G$ состоит из точек из $G$, таких что:
\[ 
\inter G = \{ \bx \in G \; | \; \exists \varepsilon > 0, B(\bx, \varepsilon) \subset G\},
\]
где $B(\bx, \varepsilon) = \{ \by \; | \; \| \bx - \by \| \leq \varepsilon \}$
\end{block}

\begin{block}{Относительная внутренность}
Относительной внутреностью множества $G$ называют следующее множество: 
\[
\relint G = \{ \bx \in G \; | \; \exists \varepsilon > 0,  B(\bx, \varepsilon) \cap \mathbf{aff} G \subseteq G \}
\]
\end{block}
Вопрос: зачем нужна концепция относительной внутренности?
\end{frame}

\begin{frame}{Примеры}

\end{frame}

\begin{frame}{Проекция точки на множество}

\end{frame}

\begin{frame}{Примеры}

\end{frame}

\begin{frame}{Опорная гиперплоскость}

\end{frame}

\begin{frame}{Примеры}

\end{frame}

\begin{frame}{Отделимость выпуклых множеств}

\end{frame}

\begin{frame}{Примеры}

\end{frame}

\begin{frame}{Резюме}
\begin{itemize}
\item Внутренность и относительная внутренность выпуклого множества
\item Проекция точки на множство
\item Опорная гиперплоскость
\item Отделимость выпуклых множеств
\end{itemize}
\end{frame}



\end{document}